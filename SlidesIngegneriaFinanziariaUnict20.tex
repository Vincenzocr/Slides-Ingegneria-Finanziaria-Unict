\documentclass{beamer}
\mode<presentation> {
\usetheme{Madrid}
}
\usepackage{graphicx} % Allows including images
\usepackage{booktabs}
\RequirePackage[utf8x]{inputenc} % Allows the use of \toprule, \midrule and \bottomrule in tables
\usepackage{amsmath}
\usepackage{dsfont}
\usepackage{amsthm}
\newtheorem{remark}{Remark}


%----------------------------------------------------------------------------------------
%	TITLE PAGE
%----------------------------------------------------------------------------------------

\title{Introduzione all'Ingegneria Finanziaria}
\author{\textbf{Vincenzo Eugenio Corallo}}


\institute[DSE Sapienza] 
{	
	\footnotesize{vincenzo.corallo@uniroma1.it}\\
	\bigskip
	\bigskip
	\includegraphics[width=20 mm]{Uniroma1.png} \\ 	
	\bigskip
	Doctoral School of Economics (DSE)
}
\date{\today} 

\begin{document}

\begin{frame}
	\titlepage % Print the title page as the first slide
\end{frame}



%----------------------------------------------------------------------------------------
%	PRESENTATION SLIDES
%----------------------------------------------------------------------------------------


\begin{frame}
\frametitle{Table of contents}
	\footnotesize{\tableofcontents}
\end{frame}

\section{Richiami di Finanza Matematica}
\begin{frame}
\frametitle{Set-up standard}
	Dato un'insieme $\Omega$, anche detto spazio di degli eventi possibili, una famiglia $\mathcal{F}$ di suoi sottoinsiemi è un'$\sigma$\textit{-algebra} per $\Omega$ se essa è chiusa\footnote{In matematica, si dice che un'operazione $\#$ definita su un insieme non vuoto $X$, verifica la proprietà chiusura di se:
		$$ \forall x,y \in X, \quad x\#y \in X$$	
	ovvero se essa è interna su $X$. Alternativamente si dice che l'insieme $X$ è chiuso rispetto all'operazione $\#$. } rispetto alle operazioni di:
	\begin{itemize}
		\item \textbf{unione}: se una famiglia numerabile di insiemi $\{A_i\}_{i \in \mathbb{N}} \in \mathcal{F}$ allora anche la loro unione $A = \bigcup^{\infty}_{i=1}A_i \in \mathcal{F}$
		\item \textbf{complementazione}: se $A \in \mathcal{F} \rightarrow A^C \in \mathcal{F} $
	\end{itemize} 
	$(\Omega, \mathcal{F})$ si dice \textbf{spazio misurabile}. Se $\Omega = \mathbb{R}$ (insieme dei numeri reali) e $B$ è la $\sigma$-algebra (detta di Borel), generata dagli intevalli aperti di $\mathbb{R}$, $(\mathbb{R}, B)$ si dice spazio di Borel e gli elementi di $B$ si dicono insiemi di Borel.
\end{frame}

\begin{frame}
\frametitle{Misure di probabilità}
	Una misura di probabilità $\mathcal{P}$ è una funzione a valori reali non negativi tale che:
	\begin{align}
	\begin{split}
		\mathcal{P}(A) &\in [0,1] \quad \forall A \in \mathcal{F} \\
		\mathcal{P}(\cup_j A_j)&=\sum_{j}\mathcal{P}(A_j) \quad \forall A_i\cap A_j =\emptyset \quad i\neq j\\
		\mathcal{P}(\Omega)&=1
	\end{split}
	\end{align}
	Una tripla $(\Omega, \mathcal{F},\mathcal{P})$ si dice spazio di probabilità con filtrazione, in cui $\Omega$ è lo spazio degli eventi elementari, $\omega \in \Omega$, $\mathcal{F}$ è un'algebra di $\Omega$ e $\mathcal{P}$ è una misura di probabilità detta naturale.
	\\~\\
	Un'affermazione è "quasi sicura", e si scrive $\mathcal{P}-a.s.$, se l'insieme $G$ in cui è falsa ha probabilità nulla: $\mathcal{P}(G)=0$.
	\\~\\
	Due o più eventi $A_i$ sono \textbf{stocasticamente indipendenti} se:
	$\mathcal{P}\left(\cap_i A_i \right) = \prod_{i} \mathcal{P}(A_i)$

	
\end{frame}

\begin{frame}
\frametitle{Variabili aleatorie}
	 Una variabile aleatoria (v.a.) $X$ è una funzione da $\Omega$ in $\mathbb{R}$ tale che per ogni $a \in \mathbb{R}$, l'insieme $\{\omega \in\Omega: X(\omega)\leq a\}	\in \mathcal{F}$, è un evento.
	 \\~\\
	 La funzione $F_X(a) = \mathcal{P}(\omega 	\in \Omega : X(\omega) \leq a)$ da $\mathbb{R}$ in $[0,1]$ si dice \textbf{funzione di ripartizione} di $X$.
	 \\~\\
	 La v.a. $X$ è \textbf{$\mathcal{F}$-misurabile} se l'immagine inversa degli intervalli aperti di $\mathbb{R}$ appartiene a $\mathcal{F}$ vale a dire se $\{\omega\in \Omega : X(\omega)\in I \} \in \mathcal{F}$, per ogni $I \in B$, spazio di Borel.


\end{frame}
\begin{frame}
	 La v.a. $X$ è \textbf{integrabile} (rispettivamente quadrato integrabile) se $\mathbb{E}(|X|)\leq \infty$ (se $\mathbb{E}(|X|^2)\leq \infty$ ).
	\\~\\
	Una v.a. ha media finita se e solo se è integrabile.
	\\~\\
	Date due v.a. $X$ e $Y$ e un evento $H= \{\omega \in \Omega:Y(\omega) \in D \}$ a probabilità non nulla, si definisce \textbf{probabilità condizionata} di $X$ dato $H$ come:
		$$	\mathcal{P}(X(\omega)\in A \mid Y(\omega)\in D)= \frac{\mathcal{P}(\{\omega \in \Omega:X(\omega) \in A)\} \cap H)}{\mathcal{P}(H)}$$
\end{frame}

\begin{frame}
\frametitle{Processi stocastici}
		Un \textbf{processo stocastico }$(X(\omega,t), t \in T)$, scritto anche $X_t(\omega) $ è una famiglia di variabili aleatorie indicizzate al tempo, con $t$ insieme discreto $\{t_0, t_1,..., t_n\}$ o continuo $[0,T]$. $X(.,t)$ per $t$ dato è una variabile aleatoria, mentre $X(\omega,.)$ per $\omega$ dato è una funzione del tempo detta traiettoria o sentiero campionario.
			\\~\\
	Un p.s. si dice di classe
\end{frame}
\begin{frame}
	\Huge{\centerline{Grazie per l'attenzione!}}
\end{frame}




\end{document}