\documentclass[]{article}
\usepackage[utf8]{inputenc}
\usepackage{graphicx}
\usepackage{amsmath}
\usepackage{amsfonts}
\usepackage{mathtools}
\usepackage{fancyhdr}
\usepackage{authblk}
\RequirePackage{natbib}
\RequirePackage{hyperref} 
\RequirePackage{amssymb}
\usepackage{dsfont}
\usepackage{amsthm}
\pagestyle{fancy}
\fancyhf{}	

\title{\textbf{Introduzione all'Ingegneria Finanziaria} 
	\newline
\textit{Corso professionalizzante per studenti di Laurea Magistrale}}
\author{Vincenzo Eugenio Corallo}
\date{}
\begin{document}

\maketitle

%\begin{abstract}
%
%\end{abstract}

\begin{itemize}
	\item \textbf{Giovedì 7 Maggio 2020, ore 10:00-12:30} (Aula da definire) - Richiami di Probabilità e Calcolo Stocastico  
	
	\item \textbf{Giovedì 7 Maggio 2020, ore 14:30-17:00} (Aula da definire) - Prima parte di richiami di Finanza Matematica
	
	\item \textbf{Venerdì 8 Maggio 2020, ore 10:00-12:30} - (Aula da definire) - Seconda parte di richiami di Finanza Matematica
		
	\item \textbf{Venerdì 8 Maggio 2020, ore 14:30-17:00} - (Aula da definire) - Terza parte di richiami di Finanza Matematica
	
	\item \textbf{Giovedì 28 Maggio 2020, ore 10:00-12:30} - (Aula da definire) -  Configurazione PyCharm e Anaconda. Introduzione al sistema di controllo versione Git. Introduzione al linguaggio Python
	 
	\item \textbf{Giovedì 28 Maggio 2020, ore 14:30-17:00} - (Aula da definire) Introduzione al codice sorgente C++ di QuantLib e alle interfacce di tale libreria per Python e Excel
	 
	\item \textbf{Venerdì 29 Maggio 2020, ore 10:00-12:30} - (Aula da definire) - Esercitazione QuantLib su Python
	
	\item \textbf{Venerdì 29 Maggio 2020, ore 14:30-17:00} (Aula da definire) - Review di un project work di gruppo
	
\end{itemize}
	Il Corso professionalizzante si pone i seguenti obiettivi:
	\begin{itemize}
		\item richiamare in modo rigoroso i fondamenti teorici della modellistica matematica per la finanza
		\item fornire una disamina delle principali convenzioni di mercato che giocano un ruolo rilevante nel pricing degli strumenti finanziari
		\item fornire gli elementi di conoscenza di base del linguaggio di programmazione Python
		
		\item introdurre la QuantLib, la più usata libreria finanziaria \textit{open source} in C++ per il pricing di strumenti finanziari
		\item coinvolgere gli studenti in una simulazione di  \textit{workflow} industriale di sviluppo condiviso di un progetto
	\end{itemize}

	Il corso è professionalizzante. E’ rivolto a studenti di Lauree Magistrali con solide conoscenze di teoria delle probabilità ed in particolare di calcolo stocastico nonchè conoscenza di base di modelli di pricing e linguaggi di programmazione.

\end{document}
